\documentclass{resume} % Use the custom resume.cls style
\usepackage{outlines}
\usepackage[left=0.4 in,top=0.4in,right=0.4 in,bottom=0.4in]{geometry} % Document margins
\newcommand{\tab}[1]{\hspace{.2667\textwidth}\rlap{#1}}
\newcommand{\itab}[1]{\hspace{0em}\rlap{#1}}
% \newcounter{cenumi}
% \newcounter{cenumisaved}
% \setcounter{cenumisaved}{0}
% \newcommand{\labelcenumi}{\arabic{cenumi}.}
% \newenvironment{cenumeratei}%
%   {\begin{list}{$\bullet$}{\usecounter{cenumi}\itemsep=-0.25em\vspace{-0.25em}}%
% \setcounter{cenumi}{\value{cenumisaved}}}%
%   {\setcounter{cenumisaved}{\value{cenumi}}%
% \end{list}}
% \renewcommand{\outlinei}{cenumeratei}

\newenvironment{cenumeratei}%
{\begin{list}{$\bullet$}{\itemsep=-0.25em\vspace{-0.25em}}%
  }%
    {%
  \end{list}}
\renewcommand{\outlinei}{cenumeratei}

\newenvironment{cenumerateii}%
  {\begin{list}{$\circ$}{\itemsep=-0.31em\vspace{-0.31em}}%
  }%
    {%
  \end{list}}
\renewcommand{\outlineii}{cenumerateii}

% Metadata
\title{Hookey Chiang's Resume}
\author{Hookey Chiang}
\date{\today\\v0.1}

% Content
\name{Tsung-Han Chiang} % Your name
% You can merge both of these into a single line, if you do not have a website.
\address{+886(932)762-686 \textbar\, Taiwan/Remote \textbar\,
\href{mailto:hookey123456@gmail.com}{hookey123456@gmail.com} \textbar\,
\href{https://www.linkedin.com/in/hookey-chiang/}{https://www.linkedin.com/in/hookey-chiang}}

\begin{document}

%------------------------------------------------------------------------------

\begin{rSection}{SUMMARY} % TODO
Being a continuous learner with a background in analyzing system performance.
Passionate about identifying problems and simplifying complex solutions.
Focusing on succinct communication.
%  Managed and deployed Linux system with 3+ years of experience, got hands-on experience in user accounts management,
% system performance monitoring, scalable virtual network deployment, and vulnerability scanning. Looking to apply
% technical skills in developing secure IT systems and contributing to a company’s security initiatives.
\end{rSection} % SUMMARY

%------------------------------------------------------------------------------

% \begin{rSection}{SKILLS}
%
% \begin{tabular}{ @{} >{\bfseries}l @{\hspace{6ex}} l }
% Technical Skills & A, B, C, D \cr
% Soft Skills & A, B, C, D\cr
% XYZ & A, B, C, D\cr
% \end{tabular}
% \end{rSection}

\begin{rSection}{WORK EXPERIENCE} % TODO
\textbf{Ubiquiti inc. \hfill Taiwan/Remote}
\\{UniFi-OS Storage Software Engineer \hfill Mar 2022 - Present}%
\begin{outline}
  \1 Delievered UniFi-NAS NPIs, including bringups, factory test plans, and service
maintenance on Debian.
    \2 Upgraded Linux from 4.19 to 5.10, handling incompatibilities with back-tracking mainline commits.
    \2 Provided technical guidance, implementing cohesive user-friendly NAS solutions involving Samba, snapshots, data backup, data integrity, system-data migration, secure network discovery; shared these specifications, documents, reproducible code snippets in Confluence with cross-teams.
  \1 Redesigned the fs layout to reduce needless writes by 50\% and address leakage of internal data.
  \1 Tuned btrfs configs on atime, space\_cache, qgroup usage to achieve
  better 8\% read and 5\% write.
  \1 Quantified and worked-around hours-long subvolume deletions causing btrfs unresponsive.
  \1 Brought system stability and performance improvements to all products with UniFi-OS.
  \1 Reduced overheads of a python cli by 65\% of CPU, 50\% of memory, then
  resulting in 15\%-to-50\% less-load-average system. Then introduced a gRPC state exporter for subscription and streaming.
  \1 Resolved an OOM due to socket buffers on a 64kB page-sized system,
  reducing 93\% over-allocation.
  \1 Assured software quality with tools such as golang test framework, xfstest, and packetdriller.
\end{outline}
% Ubiquiti

\textbf{QNAP System inc. \hfill Taiwan}
\\{Software Engineer \hfill Jan 2018 - Feb 2022}%
\begin{outline}
  % \item Achieved X\% growth for XYZ using A, B, and C skills.
  % \item Led XYZ which led to X\% of improvement in ABC
  % \item Developed XYZ that did A, B, and C using X, Y, and Z.
  \1 Took full part in Hybridmount, a filesystem as cloud storage gateway, running on 190K of QNAP NASes as of 2021, capable of video streaming and accessible through common protocols, SMB/NFS/FTP.
  \1 Designed database schema, CLI, RPC protocol, loopback-fs cache management.
  \1 Handled cloud-, platform- variant filename encodings so as to utilize Linux VFS dcache.
  \1 Shortened duration of compiling codes from 30 minutes to less than 10 minutes.
  \1 Shortened duration of copying 50 millions of files from 60 days to less than 7 days.
  \1 Maintained software quality by composing unit tests, integration tests, FS testsuites and benchmarks, run with Jenkins and Docker. Handled less than 20 support tickets since the first official release.
  \1 Identified multiple deadlocks and data races. Improved sequential write
    by 50\% faster, random read 10\%, metadata speed 300\% with optimizations, including multi-threading, fine-grained locking, VFS caching, file read-ahead, filename hashing, database index.
  \1 Realized features in deduplication filesystem, write-back cache and
    chunk-defragmentation. Reworked read/write operations with 50\% less lines
    of code and fixed abnormal data shown at chunk edges.
  \1 Analyzed and created POC of a POSIX-compliant cloud-tiering file system with FUSE, IPFS in golang.
\end{outline}% QNAP
\end{rSection}% EXPERIENCE

%------------------------------------------------------------------------------

\begin{rSection}{EDUCATION}
\textbf{National Taiwan University \hfill Taipei, Taiwan}
\\M.S. in Computer Science and Information Engineering \hfill Sep 2014 - Jun
2016%
\begin{outline}
  \1 \textbf{GPA}: 3.84/4.30
  \1 \textbf{Thesis}: Program Analysis with A Loop-function-based Tracing Tool
  on Virtual Platform.
    \2 Rewarded as the best paper of 2017 ACM Research in Adaptive and Convergent Systems.
\end{outline}

\textbf{National Chiao Tung University \hfill Hsinchu, Taiwan}
\\B.S. in Electrical Engineering and Computer Science \hfill Sep 2010 - Jun
2014%
\begin{outline}
  \1 \textbf{Grade}: 90.2\%
  \1 \textbf{Award}: Recieved 4 times of the Academic Achievement Award.
\end{outline}

\textbf{University of Illinois at Urbana-Champaign \hfill Champaign, IL}
\\Exchange Student in Computer Science \hfill Aug 2013 - Dec 2013%

\end{rSection}% Education

%------------------------------------------------------------------------------

\begin{rSection}{PROJECTS}
% \textbf{Program Analysis with A Loop-function-based Tracing Tool on
%   Virtual Platform.}
% \begin{outline}
% {Rewarded as the best paper of 2017 ACM Research in Adaptive and Convergent Systems.
% This work leveraged QEMU to collect detailed program performance metrics from
% hardware and software across call context trees, which summarizes the program
% execution order of branches.}
% \end{outline}

\textbf{Phase-based Profiling and Performance Prediction with Timing
  Approximate Simulators.}
\begin{outline}
  \1 Achieved cycle-approximate simulation and to speed up design space
  exploration on QEMU.
  %, sponsored by the Ministry of Science and Technology of Taiwan and MediaTek Inc.
  \1 Role- Was to implement Just-in-Time Model Selection, dedicated to speeding
  up the architectural simulation by detecting repeating patterns and converting
  them into faster timing models.
\end{outline}

\textbf{Thread Cluster Memory with Request Reordering on the UTAH
  Simulated Memory.}
\begin{outline}
  \1 Proposed memory scheduling algorithm, which dynamically groups threads into
  two clusters based on memory intensity, outperforming the
  first-come-first-serve about 10\% on execution time, 19\% on fairness, and
  21\% on energy delay product. And got first place in the NTU-in-course contest.
  \1 Role- Was to make a research on efficient memory schedulers and factor out performance implications.
\end{outline}

\textbf{CodeXL-aided Steiner Tree Acceleration on Heterogeneous System
  Architecture.}
\begin{outline}
  \1 Optimized GPU kernels with CodeXL GPU profiler, increasing the usage of
  high-speed general purpose registers, reducing slow-speed memory usage, and
  ultimately improving performance by 40\%.
\end{outline}

\textbf{Pharmacy POS System}
\begin{outline}
  \1 Built a POS system based on the Electron framework, for pharmacists to
  receive prescriptions and report results to the Bureau of National Health Insurance(NHI).
  \1 Role- Was to figure out data spec of NHI and implement card-agent on Windows to
communicate NHI cards with APIs and to interact with web-based POS.
\end{outline}

\end{rSection} % PROJECTS

%----------------------------------------------------------------------------------------

\begin{rSection}{LEADERSHIP}
\textbf{Chair of residents’ committee}. \hfill \textbf{Jul 2021 - Jun 2023}
\begin{outline}
  \1 Achieved to expedite administrative tasks and reduce ramp-up time for
  future successors by establishing standard operating procedures, introducing
  paperless document management and implementing templates.
  \1 Optimized community regulations regarding meeting quorum and the election
  system.
\end{outline}

\textbf{Chair of EECS department student council}. \hfill \textbf{Sep 2012 - Jun 2013}
\begin{outline}
  \1 Organized after-school tutoring for 6 students struggling with learning programming.
  \1 Hold an orientation event where professors and seniors shared aspects
  of university life with 30 freshmen.
\end{outline}
\end{rSection} % Leadership

\end{document}
